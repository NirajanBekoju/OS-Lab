              
\documentclass[journal,onecolumn]{IEEEtran}
\usepackage{multicol}
\setlength{\parindent}{0pt}

\ifCLASSINFOpdf
 
\else
 
\fi

% correct bad hyphenation here
\hyphenation{op-tical net-works semi-conduc-tor}


\begin{document}

\title{LAB 4 : File Management System in Operating System}

\author{
  \begin{multicols}{2}
    \centering
    Nirajan Bekoju\\
    PUL076BCT039\\
    076bct039.nirajan@pcampus.edu.np\\
    \columnbreak
    Nishant Luitel\\
    PUL076BCT041 \\
    076bct041.nishant@pcampus.edu.np \\
  \end{multicols}
  \begin{multicols}{2}
    \centering
    Nabin Da Shrestha\\
    PUL076BCT037 \\
    076bct037.nabin@pcampus.edu.np \\
    \columnbreak
    Prakash Chaulagain \\
    PUL076BCT045 \\
    076bct045.prakash@pcampus.edu.np
  \end{multicols}
}

% The paper headers
\markboth{LAB 4 : File Management System in Operating System}{}

% make the title area
\maketitle

\IEEEpeerreviewmaketitle

\section{Introduction}
A file management system in an operating system is responsible for managing the storage and retrieval of files on a disk or other storage device. It provides an interface to create, read, write, delete and modify files.

In a file management system, a file is a named collection of data that is stored in a disk or other storage device. Each file is identified by a unique name, which is used to locate it in the file system.

Reading a file in C programming involves opening the file with the fopen() function, which returns a pointer to a FILE structure. This structure contains information about the file, such as its name and position in the file. After opening the file, data can be read from it using functions such as fread() or fgets().

To write to a file in C programming, the file must be opened in write mode using the fopen() function with the "w" parameter. Data can then be written to the file using functions such as fwrite() or fputs().

Updating a file involves opening the file in update mode using the fopen() function with the "r+" or "w+" parameter. This allows both reading and writing to the file. Data can be read from the file using functions such as fread() or fgets(), and data can be written to the file using functions such as fwrite() or fputs().

In addition to basic file operations, a file management system may also provide features such as file locking, file permissions, and file compression. These features help ensure the security and efficiency of the file system.



\section{File Creation}
\textbf{Aim : } To write a C program to create a file.

\textbf{Algorithm : }
\begin{list}{}{}
  \item STEP 1 : Start the program.
  \item STEP 2 : Create the file using create function and assign a variable to it.
  \item STEP 3 : If the value of the variable is less then print file cannot be created ,Otherwise print file is created.
  \item STEP 4 : Stop the program.
\end{list}

\textbf{Program : }
\begin{verbatim}
  #include <stdio.h>

  #include<sys/types.h>
  #include<sys/stat.h>
  
  int main(){
      int id;
      // create return 0 on success and -1 on error
      if(id = creat("z.txt", 0) == -1){
          printf("cannot create the file\n");
          exit(1);
      }
      else{
          printf("file is created");
          exit(1);
      }
  }
\end{verbatim}

\textbf{Ouput :}
\begin{verbatim}
  file is created

  $ ls 
  z.txt
\end{verbatim}



\section{Writing Into A File}
\textbf{Aim : } To write a C program to write the data into a file.

\textbf{Algorithm : }
\begin{list}{}{}
  \item STEP 1 : Get the data from the user. 
  \item STEP 2 : Open a file.
  \item STEP 3 : Write the data from the file. 
  \item STEP 4 : Get the data and update the file.
\end{list}

\textbf{Program : }
\begin{verbatim}
  #include <stdio.h>

  int main(){
      char str[100];
      FILE *fp;
      printf("Enter the string : ");
      gets(str);
      
      fp = fopen("file1.dat", "w+");
      while (!feof(fp))
      {
          fscanf(fp, "%s", str);
      }
      fprintf(fp, "%s", str);
      fclose(fp);
  }
\end{verbatim}

\textbf{Output:}
\begin{verbatim}
  Enter the string : this is file management system in os.

  $ cat file1.dat 
  this is file management system in os.
\end{verbatim}

\section{Reading From a File}
\textbf{Aim :} To create and read data from the file.

\textbf{Algorithm :}
\begin{list}{}{}
  \item STEP 1 : Get the data from the user.
  \item STEP 2 : Open a file.
  \item STEP 3 : Read from the file. 
  \item STEP 4 : Close the file.
\end{list}


\textbf{Program :}
\begin{verbatim}
  #include <stdio.h>

  int main(){
      char str[100];
      FILE *fp;
      fp = fopen("file1.dat", "r");
      while (!feof(fp))
      {
          fscanf(fp, "%s", str);
          printf("%s ", str);
      }
      printf("\n");
      fclose(fp);
  }
\end{verbatim}

\textbf{Output :}
\begin{verbatim}
  this is file management system in os.
\end{verbatim}




\section{Conclusion}
The implementation of a file management system in the operating system using C programming language has been successful in providing the fundamental functionalities of file creation, reading, and writing. The program allows users to create a file with a given name and extension, read data from an existing file, and write data into a file. These basic operations form the backbone of any file management system and are essential for handling data in a computer system. The implementation serves as a good starting point for developing more complex file systems that can handle larger amounts of data and provide additional features such as file deletion, file permissions, and file sharing.

\end{document}