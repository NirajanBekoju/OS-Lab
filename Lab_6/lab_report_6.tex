              
\documentclass[journal,onecolumn]{IEEEtran}
\usepackage{multicol}
\setlength{\parindent}{0pt}

\ifCLASSINFOpdf
 
\else
 
\fi

% correct bad hyphenation here
\hyphenation{op-tical net-works semi-conduc-tor}


\begin{document}

\title{LAB 6 : Process Scheduling, Communication and Synchronization in OS}

\author{
  \begin{multicols}{2}
    \centering
    Nirajan Bekoju\\
    PUL076BCT039\\
    076bct039.nirajan@pcampus.edu.np\\
    \columnbreak
    Nishant Luitel\\
    PUL076BCT041 \\
    076bct041.nishant@pcampus.edu.np \\
  \end{multicols}
  \begin{multicols}{2}
    \centering
    Nabin Da Shrestha\\
    PUL076BCT037 \\
    076bct037.nabin@pcampus.edu.np \\
    \columnbreak
    Prakash Chaulagain \\
    PUL076BCT045 \\
    076bct045.prakash@pcampus.edu.np
  \end{multicols}
}

% The paper headers
\markboth{LAB 6 : Process Scheduling, Communication and Synchronization in OS}%
{Shell \MakeLowercase{\textit{et al.}}: Bare Demo of IEEEtran.cls for IEEE Journals}

% make the title area
\maketitle

\IEEEpeerreviewmaketitle

\section{Introduction}
Process scheduling, communication, and synchronization are essential concepts in operating systems that deal with managing multiple processes and ensuring that they run efficiently and effectively.

Process scheduling is the process of allocating CPU time to the various processes in a computer system. In modern operating systems, multiple processes may run simultaneously, and the operating system must schedule and manage their execution to ensure that the system runs efficiently. The scheduler determines which process should run next based on various factors, such as priority, the amount of time a process has already used, and the available resources.

Communication between processes is necessary for a variety of reasons, including inter-process communication (IPC), where different processes need to exchange data or signals to achieve some common goal. Processes can communicate with each other through shared memory, message passing, or pipe mechanisms, depending on the operating system and the specific needs of the processes.

Synchronization is the process of coordinating the execution of multiple processes to ensure that they do not interfere with each other or create conflicts while accessing shared resources. Synchronization is essential to avoid race conditions, deadlock, and other issues that can arise when multiple processes attempt to access shared resources simultaneously. Techniques like semaphores, mutex, and conditional variables are used to achieve synchronization among processes.

In summary, process scheduling, communication, and synchronization are all critical components of operating system design, as they ensure that the various processes in a system can work together efficiently, without conflicts or interference, while utilizing the available resources optimally.


\section{First Come First Serve}
\textbf{Aim : } To write a C program to implement the CPU scheduling algorithm for First Come First Serve

\textbf{Problem Description : }
Cpu scheduler will decide which process should be given the CPU for its execution.For this it use different algorithm to choose among the process. one among that algorithm is fcfs algorithm.

In this algorithm the process which arrive first is given the cpu after finishing its request only it will allow cpu to execute other process.

\textbf{Algorithm : }
\begin{list}{}{}
  \item Step 1 : Create the number of process.
  \item Step 2 : Get the ID and Service time for each process.
  \item Step 3 : Initially, Waiting time of first process is zero and Total time for the
        first process is the starting time of that process.
  \item Step 4 : Calculate the Total time and Processing time for the remaining
        processes.
  \item Step 5 : Waiting time of one process is the Total time of the previous
        process.
  \item Step 6 : Total time of process is calculated by adding Waiting time and Service
        time.
  \item Step 7 : Total waiting time is calculated by adding the waiting time for lack
        process.
  \item Step 8 : Total turn around time is calculated by adding all total time of each
        process.
  \item Step 9 : Calculate Average waiting time by dividing the total waiting time by
        total number of process.
  \item Step 10 : Calculate Average turn around time by dividing the total time by
        the number of process.
  \item Step 11 : Display the result.
\end{list}

\textbf{Program : }
\begin{verbatim}
  #include <stdio.h>
  #include <unistd.h>
  
  // Turnaround Time = completion of a process - submission of a process
  // Waiting Time = turnaround time - burst time
  
  // Function to find the waiting time for all processes
  int waitingTime(int proc[], int n, int burst_time[], int wait_time[]){
      wait_time[0] = 0;
      for (int i = 1; i < n; i++){
          wait_time[i] = wait_time[i - 1] + burst_time[i - 1];
      }
      return 0;
  }
  
  // calculating the turn around time
  int turnAroundTime(int proc[], int n, int burst_time[], int wait_time[], int tat[]){
      for (int i = 0; i < n; i++){
          tat[i] = wait_time[i] + burst_time[i];
      }
      return 0;
  }
  
  // calculate the average time
  int avgTime(int proc[], int n, int burst_time[]){
      int wait_time[n], tat[n], total_wt = 0, total_tat = 0;
      int i;
  
      waitingTime(proc, n, burst_time, wait_time);
      turnAroundTime(proc, n, burst_time, wait_time, tat);
  
      printf("Processes  Burst   Waiting Turn around \n");
      // calculate total waiting time and total turn around time
      for(i = 0; i < n; i++){
          total_wt += wait_time[i];
          total_tat += tat[i];
          printf("%d\t  %d\t\t %d \t%d\n", i+1, burst_time[i], wait_time[i], tat[i]);
      }
  
      printf("Average waiting time = %f\n", (float)total_wt / (float)n);
      printf("Average turn around time = %f\n", (float)total_tat / (float)n);
      return 0;
  }
  
  int main(){
      int proc[] = {1, 2, 3};
      int n = sizeof proc / sizeof proc[0];
      int burst_time[] = {5, 8, 12};
      avgTime(proc, n, burst_time);
      return 0;
  }
\end{verbatim}


\textbf{Ouput :}
\begin{verbatim}
  Processes  Burst   Waiting Turn around 
  1         5              0      5
  2         8              5      13
  3         12             13     25
  Average waiting time = 6.000000
  Average turn around time = 14.333333
\end{verbatim}



\section{Shortest Job First}
\textbf{Aim : } To write a C program to implement the CPU scheduling algorithm for Shortest job first.

\textbf{Problem Description : }
Cpu scheduler will decide which process should be given the CPU for its execution. For this it use different algorithm to choose among the process. one among that algorithm is sjf algorithm.

In this algorithm the process which has less service time given the cpu after finishing its request only it will allow cpu to execute next other process.

\textbf{Algorithm : }
\begin{list}{}{}
  \item Step 1 : Get the number of process.
  \item Step 2 : Get the id and service time for each process.
  \item Step 3 : Initially the waiting time of first short process as 0 and total time of
        first short is process the service time of that process.
  \item Step 4 : Calculate the total time and waiting time of remaining process.
  \item Step 5 : Waiting time of one process is the total time of the previous process.
  \item Step 6 : Total time of process is calculated by adding the waiting time and
        service time of each process.
  \item Step 7 : Total waiting time calculated by adding the waiting time of each
        process.
  \item Step 8 : Total turn around time calculated by adding all total time of
        each process.
  \item Step 9 : Calculate average waiting time by dividing the total waiting time by
        total number of process.
  \item Step 10 : Calculate average turn around time by dividing the total waiting time
        by total number of process.
  \item Step 11 : Display the result.
\end{list}

\textbf{Program : }
\begin{verbatim}
  #include <stdio.h>
  #include <unistd.h>
  
  // Turnaround Time = completion of a process – submission of a process
  // Waiting Time = turnaround time – burst time
  
  // Function to find the waiting time for all processes
  int waitingTime(int proc[], int n, int burst_time[], int wait_time[]){
      wait_time[0] = 0;
      for (int i = 1; i < n; i++){
          wait_time[i] = wait_time[i - 1] + burst_time[i - 1];
      }
      return 0;
  }
  
  // calculating the turn around time
  int turnAroundTime(int proc[], int n, int burst_time[], int wait_time[], int tat[]){
      for (int i = 0; i < n; i++){
          tat[i] = wait_time[i] + burst_time[i];
      }
      return 0;
  }
  
  // calculate the average time
  int avgTime(int proc[], int n, int burst_time[]){
      int wait_time[n], tat[n], total_wt = 0, total_tat = 0;
      int i;
  
      waitingTime(proc, n, burst_time, wait_time);
      turnAroundTime(proc, n, burst_time, wait_time, tat);
  
      printf("ProcessesId  Burst   Waiting Turn around \n");
      // calculate total waiting time and total turn around time
      for(i = 0; i < n; i++){
          total_wt += wait_time[i];
          total_tat += tat[i];
          printf("%d\t\t  %d\t %d \t%d\n", proc[i], burst_time[i], wait_time[i], tat[i]);
      }
  
      printf("Average waiting time = %f\n", (float)total_wt / (float)n);
      printf("Average turn around time = %f\n", (float)total_tat / (float)n);
      return 0;
  }
  
  void swap(int* xp, int* yp)
  {
      int temp = *xp;
      *xp = *yp;
      *yp = temp;
  }
  
  int shortestJobFirst(int proc[], int n, int burst_time[]){
      int shortest_process_queue[n];
      for (int i = 0; i < n - 1; i++){
          // last i elements are already in place
          for (int j = 0; j < n - i - 1; j++){
              if (burst_time[j] > burst_time[j + 1]){
                  swap(&burst_time[j], &burst_time[j + 1]);
                  swap(&proc[j], &proc[j + 1]);
              }
          }
      }
      avgTime(proc, n, burst_time);
      return 0;
  }
  
  int main(){
      int proc[] = {5, 2, 3, 4, 100, 457};
      int n = sizeof proc / sizeof proc[0];
      int burst_time[] = {8, 5, 12, 5, 100, 1};
      shortestJobFirst(proc, n, burst_time);
      return 0;
  }
  
\end{verbatim}


\textbf{Ouput :}
\begin{verbatim}
  ProcessesId  Burst   Waiting Turn around 
  457               1      0      1
  2                 5      1      6
  4                 5      6      11
  5                 8      11     19
  3                 12     19     31
  100               100    31     131
  Average waiting time = 11.333333
  Average turn around time = 33.166668
\end{verbatim}



\section{Round Robin}
\textbf{Aim : } To write a C program to simulate the CPU scheduling algorithm for round robin.

\textbf{Problem Description : }
CPU scheduler will decide which process should be given the CPU for its execution .For
this it use different algorithm to choose among the process .one among that algorithm is Round
robin algorithm.
In this algorithm we are assigning some time slice .The process is allocated according to
the time slice ,if the process service time is less than the time slice then process itself will release
the CPU voluntarily .The scheduler will then proceed to the next process in the ready queue .If
the CPU burst of the currently running process is longer than time quantum ,the timer will go off
and will cause an interrupt to the operating system .A context switch will be executed and the
process will be put at the tail of the ready queue.

\textbf{Algorithm : }
\begin{verbatim}
  Step 1: Initialize all the structure elements
  Step 2: Receive inputs from the user to fill process id,burst time and arrival time. 
  Step 3: Calculate the waiting time for all the process id.
      i) The waiting time for first instance of a process is calculated as:

          a[i].waittime=count + a[i].arrivt
      
      ii) The waiting time for the rest of the instances of the process is 
          calculated as:
          
          a) If the time quantum is greater than the remaining burst time
          then waiting time is calculated as:a[i].waittime=count + tq
          
          b) Else if the time quantum is greater than the remaining burst time 
          then waiting time is calculated as:
            
              a[i].waittime=count - remaining burst time
  
  Step 4: Calculate the average waiting time and average turnaround time
  Step 5: Print the results of the step 4.  
\end{verbatim}

\textbf{Program : }
\begin{verbatim}
  #include <stdio.h>

  void main()
  {
      // initlialize the variable name
      int i, NOP, sum = 0, count = 0, y, quant, wt = 0, tat = 0;
      int at[10], bt[10], temp[10];
      float avg_wt, avg_tat;
      printf(" Total number of process in the system: ");
      scanf("%d", &NOP);
      y = NOP; 
  
      // Input arrival time, burst time and time quantum
      // Use for loop to enter the details of the process 
      // like Arrival time and the Burst Time
      for (i = 0; i < NOP; i++)
      {
          printf("\n Enter the Arrival and Burst time of the Process[%d]\n", i + 1);
          printf(" Arrival time is: "); 
          scanf("%d", &at[i]);
          printf(" \nBurst time is: "); 
          scanf("%d", &bt[i]);
          // store the burst time in temp array
          temp[i] = bt[i]; 
      }
  
      // Accept the Time quantum
      printf("\nEnter the Time Quantum for the process: ");
      scanf("%d", &quant);
      
      // Display the process No, burst time, Turn Around Time and the waiting time
      printf("\n Process No \t\t Burst Time \t\t TAT \t\t Waiting Time ");
      for (sum = 0, i = 0; y != 0;)
      {
          if (temp[i] <= quant && temp[i] > 0) // define the conditions
          {
              sum = sum + temp[i];
              temp[i] = 0;
              count = 1;
          }
          else if (temp[i] > 0)
          {
              temp[i] = temp[i] - quant;
              sum = sum + quant;
          }
          if (temp[i] == 0 && count == 1)
          {
              y--; 
              printf("\nProcess No[%d] \t\t %d\t\t\t\t %d\t\t\t %d", i + 1, bt[i], 
                      sum - at[i], sum - at[i] - bt[i]);
              wt = wt + sum - at[i] - bt[i];
              tat = tat + sum - at[i];
              count = 0;
          }
  
          if (i == NOP - 1)
          {
              i = 0;
          }
          else if (at[i + 1] <= sum)
          {
              i++;
          }
          else
          {
              i = 0;
          }
      }
      // represents the average waiting time and Turn Around time
      avg_wt = (float)wt / NOP;
      avg_tat = (float)tat / NOP;
      printf("\n Average Turn Around Time: \t%f", avg_wt);
      printf("\n Average Waiting Time: \t%f", avg_tat);
  }
\end{verbatim}


\textbf{Ouput :}
\begin{verbatim}
  Total number of process in the system: 4

  Enter the Arrival and Burst time of the Process[1]
  Arrival time is: 5
  
  Burst time is: 12
 
  Enter the Arrival and Burst time of the Process[2]
  Arrival time is: 0
  
  Burst time is: 8
 
  Enter the Arrival and Burst time of the Process[3]
  Arrival time is: 3
  
  Burst time is: 15
 
  Enter the Arrival and Burst time of the Process[4]
  Arrival time is: 13
  
  Burst time is: 3 
 
 Enter the Time Quantum for the process: 5 
 
  Process No              Burst Time              TAT             Waiting Time 
 Process No[4]            3                               5                       2
 Process No[2]            8                               26                      18
 Process No[1]            12                              28                      16
 Process No[3]            15                              35                      20
\end{verbatim}



\section{Priority Scheduling}
\textbf{Aim : } To write a C program to implement CPU scheduling algorithm for priority scheduling.

\textbf{Problem Description : }
Cpu scheduler will decide which process should be given the CPU for its execution.For this
it use different algorithm to choose among the process. one among that algorithm is fcfs
algorithm.

In this algorithm the process which arrive first is given the cpu after finishing its request
only it will allow cpu to execute other process.

\textbf{Algorithm : }
\begin{list}{}{}
  \item Step 1 : Get the number of process,burst time and priority.
  \item Step 2 : Using for loopi=0 to n-1 do step 1 to 6.
  \item Step 3 : If i=0,wait time=0,T[0]=b[0];
  \item Step 4 : T[i]=T[i-1]+b[i] and wt[i]=T[i] - b[i].
  \item Step 5 : Total waiting time is calculated by adding the waiting time for lack
        process.
  \item Step 6 : Total turn around time is calculated by adding all
        total time of each process.
  \item Step 7 : Calculate Average waiting time by dividing the total waiting time by total
        number of process.
  \item Step 8 : Calculate Average turn around time by dividing the total time by the
        number of process.
  \item Step 9 :  Display the result.
\end{list}

\textbf{Program : }
\begin{verbatim}
  #include <stdio.h>
  // Turnaround Time = completion of a process – submission of a process
  // Waiting Time = turnaround time – burst time
  
  void swap(int* xp, int* yp)
  {
      int temp = *xp;
      *xp = *yp;
      *yp = temp;
  }
  
  int main(){
      int NUM = 4;
      int process[] = {1, 2, 3, 4};
      int burst_time[] = {5, 7, 2, 3};
      int priority[] = {1, 6, 4, 5};
  
      // get the sorted array based on the priority
      for (int i = 0; i < NUM - 1; i++){
          for (int j = 0; j < NUM - i - 1; j++){
              if(priority[j] < priority[j + 1]){
                  swap(&process[j], &process[j + 1]);
                  swap(&burst_time[j], &burst_time[j + 1]);
                  swap(&priority[j], &priority[j + 1]);
              }
          }
      }
  
      // T stores the starting time of the process
      int t = 0;
      printf("Executing the process in required order\n");
      for (int i = 0; i < NUM; i++){
          printf("P%d is executed from %d to %d\n", process[i], t, t + burst_time[i]);
      }
  
      printf("\n");
      int wait_time = 0;
      int total_waiting_time = 0;
      int total_TAT_time = 0;
      printf("Process ID\t Burst Time\t Wait Time \t Turn Around Time\n");
      for (int i = 0; i < NUM; i++){
          printf("P%d \t\t\t %d\t\t %d \t\t %d\n", process[i], burst_time[i], 
                  wait_time, wait_time + burst_time[i]);
          total_waiting_time += wait_time;
          total_TAT_time += wait_time + burst_time[i];
          wait_time += burst_time[i];
      }
  
      printf("Average Waiting Time is %f \n", (float)total_waiting_time / (float)NUM);
      printf("Average Turn Around Time is %f \n", (float)total_TAT_time / (float)NUM);
  
  }
\end{verbatim}


\textbf{Ouput :}
\begin{verbatim}
  Executing the process in required order
  P2 is executed from 0 to 7
  P4 is executed from 0 to 3
  P3 is executed from 0 to 2
  P1 is executed from 0 to 5
  
  Process ID       Burst Time      Wait Time       Turn Around Time
  P2                       7               0               7
  P4                       3               7               10
  P3                       2               10              12
  P1                       5               12              17
  Average Waiting Time is 7.250000 
  Average Turn Around Time is 11.500000 
\end{verbatim}


\section{Producer Consumer Problem Using Semaphore}
\textbf{Aim : } To write a C program to implement the Producer \& consumer Problem.


\textbf{Algorithm : }
\begin{list}{}{}
  \item Step 1: The Semaphore mutex, full \& empty are initialized.
  \item Step 2: In the case of producer process
        \begin{list}{}{}
          \item i. Produce an item in to temporary variable.
          \item ii.
                If there is empty space in the buffer check the mutex value for enter into
          \item iii.
                the critical section.
                If the mutex value is 0, allow the producer to add value in the temporary
                variable to the buffer.
        \end{list}

  \item Step 3: In the case of consumer process
        \begin{list}{}{}
          \item i. It should wait if the buffer is empty
          \item ii. If there is any item in the buffer check for mutex value, if the mutex==0, remove item from buffer.
          \item iii. Signal the mutex value and reduce the empty value by 1.
          \item iv. Consume the item.
        \end{list}
  \item Step 4: Print the result
\end{list}

\textbf{Program : }
\begin{verbatim}
  #include <stdio.h>

  int mutex = 0;
  int num_full_slot = 0;
  int num_empty_slot = 4;
  int x = 0;
  
  void producer(){
      // increase the mutex value so that can't interfere 
      ++mutex;
      // increase the number of full slot
      ++num_full_slot;
      // decrease the number of empty slot
      --num_empty_slot;
      
      // Produce Item
      x++;
      printf("\nProducer produces item %d", x);
      
      // signal that the producer has completed the process and 
      // no longer requires the processor
      --mutex;
  }
  
  void consumer(){
      ++mutex;
      --num_full_slot;
      ++num_empty_slot;
      printf("\nConsumer consumes item %d", x);
      x--;
      --mutex;
  }
  
  int main(){
      int choice;
      printf("\n1. Press 1 for Producer"
             "\n2. Press 2 for Consumer"
             "\n3. Press 3 for Exit");
      
      // "critical" specifies that the code is executed by only one thread 
      // at a time i.e only one thread enters the critical section at a given time
      #pragma omp critical
  
      while (1)
      {
          printf("\nEnter your choice :  ");
          scanf("%d", &choice);
  
          switch (choice)
          {
          case 1:
              if (mutex == 0 && num_empty_slot != 0){
                  producer();
              }
              else{
                  printf("Buffer is full");
              }
              break;
          case 2:
              if(mutex == 0 && num_full_slot != 0){
                  consumer();
              }
              else{
                  printf("Buffer is empty");
              }
              break;
          case 3:
              exit(0);
              break;
          default:
              break;
          }
      }
  }
\end{verbatim}


\textbf{Ouput :}
\begin{verbatim}
  1. Press 1 for Producer
  2. Press 2 for Consumer
  3. Press 3 for Exit
  Enter your choice :  1
  
  Producer produces item 1
  Enter your choice :  1
  
  Producer produces item 2
  Enter your choice :  1
  
  Producer produces item 3
  Enter your choice :  1
  
  Producer produces item 4
  Enter your choice :  1
  Buffer is full
  Enter your choice :  2
  
  Consumer consumes item 4
  Enter your choice :  2
  
  Consumer consumes item 3
  Enter your choice :  2
  
  Consumer consumes item 2
  Enter your choice :  2
  
  Consumer consumes item 1
  Enter your choice :  2
  Buffer is empty
  Enter your choice :  2
  Buffer is empty
  Enter your choice :  3
\end{verbatim}


\section{Conclusion}
In conclusion, this lab report presented the implementation of several important concepts in operating systems, including process scheduling algorithms such as FCFS, SJF, round robin, and priority scheduling, and the synchronization mechanism of semaphores in the Producer-Consumer problem. These implementations were done using the C programming language, which allowed for a detailed understanding of the underlying concepts and algorithms. Through the process of implementing and testing these algorithms and synchronization mechanisms, we gained a deeper understanding of the challenges involved in managing and coordinating multiple processes in an operating system. Overall, this lab report provides a valuable contribution to the study of operating systems, and serves as a useful resource for those interested in learning more about these essential concepts in computer science.

\end{document}