\documentclass[lettersize,journal]{IEEEtran}
\usepackage{amsmath,amsfonts}
\usepackage{algorithmic}
\usepackage{array}
\usepackage[caption=false,font=normalsize,labelfont=sf,textfont=sf]{subfig}
\usepackage{textcomp}
\usepackage{stfloats}
\usepackage{url}
\usepackage{verbatim}
\usepackage{graphicx}
\graphicspath{{image/}} % specify the path for images

\hyphenation{op-tical net-works semi-conduc-tor IEEE-Xplore}
\def\BibTeX{{\rm B\kern-.05em{\sc i\kern-.025em b}\kern-.08em
    T\kern-.1667em\lower.7ex\hbox{E}\kern-.125emX}}
\usepackage{balance}

\newcommand\tab[1][1cm]{\hspace*{#1}}


\begin{document}
\title{Operating System Lab Report}
\author{Nirajan Bekoju 076BCT039}

\maketitle


\begin{abstract}
    This document describes the Operating System Lab Exercises
\end{abstract}
\begin{IEEEkeywords}
    Class, IEEEtran, \LaTeX, paper, style, template, typesetting.
\end{IEEEkeywords}


\section{Introduction}
\IEEEPARstart{W}{elcome} to the updated and simplified documentation to using the IEEEtran \LaTeX \ class file. The IEEE has examined hundreds of author submissions using this package to help formulate this easy to follow guide. We will cover the most commonly used elements of a journal article. For less common elements we will refer back to the ``IEEEtran\_HOWTO.pdf''.

\section{UNIX COMMANDS}
\subsection{Date Command}
\noindent This command is used to display the current date and time.

\noindent Syntax :

        \$ date 

        \$ date +\%ch
  
\noindent Options :

a = Abbreviated weekday.

A = Full weekday.

b = Abbreviated month.

B = Full month.

c = Current day and time.

C = Display the century as a decimal number.

d = Day of the month.

h = Abbreviated month day.

H = Display the hour.

L = Day of the year.

m = Month of the year.

M = Minute.

P = Display AM or PM S = Seconds

T = HH:MM:SS format u = Week of the year.

y = Display the year in 2 digit.

Y = Display the full year.

Z = Time zone . To change the format :


\subsection{Calendar}
\noindent This command is used to display the calendar of the year or the particular month of calendar year.

\noindent Syntax :

\$ cal $\langle$ year $\rangle$

\$ cal $\langle$ month $\rangle$ $\langle$ year $\rangle$


\section{Where to get the IEEEtran Templates}
\noindent 

\section{Document Class Options in IEEEtran}
\noindent At the beginning of your \LaTeX\ file you will need to establish what type of publication style you intend to use. The following list shows appropriate documentclass options for each of the types covered by IEEEtran.

There are other options available for each of these when submitting for peer review or other special requirements. IEEE recommends to compose your article in the base 2-column format to make sure all your equations, tables and graphics will fit the final 2-column format. Please refer to the document ``IEEEtran\_HOWTO.pdf'' for more information on settings for peer review submission if required by your EIC.

\section{How to Create Common Front Matter}
\noindent The following sections describe general coding for these common elements. Computer Society publications and Conferences may have their own special variations and will be noted below.

\subsection{Paper Title}
\noindent The title of your paper is coded as:

\begin{verbatim}
    \title{The Title of Your Paper}
\end{verbatim}

\noindent Please try to avoid the use of math or chemical formulas in your title if possible.

\subsection{Author Names and Affiliations}
\noindent The author section should be coded as follows:
\begin{verbatim}
\author{Masahito Hayashi 
\IEEEmembership{Fellow, IEEE}, Masaki Owari
\thanks{M. Hayashi is with Graduate School 
of Mathematics, Nagoya University, Nagoya, 
Japan}
\thanks{M. Owari is with the Faculty of 
Informatics, Shizuoka University, 
Hamamatsu, Shizuoka, Japan.}
}
\end{verbatim}

\subsection{Running Heads}
\noindent The running heads are declared by using the $\backslash${\tt{markboth}} command. There are two arguments to this command: the first contains the journal name information and the second contains the author names and paper title.


\subsection{Copyright Line}
\noindent For Transactions and Journals papers, this is not necessary to use at the submission stage of your paper. The IEEE production process will add the appropriate copyright line. If you are writing a conference paper, please see the ``IEEEtran\_HOWTO.pdf'' for specific information on how to code "Publication ID Marks".

\subsection{Abstracts}
\noindent The abstract is the first element of a paper after the $\backslash${\tt{maketitle}} macro is invoked.  The coding is simply:
\begin{verbatim}
\begin{abstract}
Text of your abstract.
\end{abstract}
\end{verbatim}
Please try to avoid mathematical and chemical formulas in the abstract.



\end{document}