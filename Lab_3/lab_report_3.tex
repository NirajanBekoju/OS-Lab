              
\documentclass[journal,onecolumn]{IEEEtran}
\setlength{\parindent}{0pt}

\ifCLASSINFOpdf
 
\else
 
\fi

% correct bad hyphenation here
\hyphenation{op-tical net-works semi-conduc-tor}


\begin{document}

\title{LAB 3 : SYSTEM CALLS}

\author{Nirajan~Bekoju \\ PUL076BCT039 \\ $1^{st}$ March, 2023}

% The paper headers
\markboth{LAB 3 : SYSTEM CALLS}%
{Shell \MakeLowercase{\textit{et al.}}: Bare Demo of IEEEtran.cls for IEEE Journals}

% make the title area
\maketitle

\IEEEpeerreviewmaketitle

\section{Introduction}
In an operating system, process creation is the process of creating a new process from an existing process. This is typically done using system calls such as fork() and exec() in the C programming language. Here's a step-by-step description of the process creation process using C:

\begin{enumerate}
  \item The parent process calls the fork() system call to create a new child process.
  \item The fork() system call creates a new process by duplicating the parent process. The child process is an exact copy of the parent process, including all the memory, data, and code.
  \item After the fork() call, the parent and child processes continue executing independently. They share the same code and data, but have separate copies of the memory.
  \item The child process can modify its own copy of the memory without affecting the parent process or any other processes.
  \item The parent process can use the returned process ID from the fork() call to identify the child process.
  \item The child process can use the exec() system call to load a new program into its memory space. This replaces the child process's memory and data with the new program and starts executing it.
  \item The parent process can wait for the child process to finish using the wait() system call. This suspends the parent process until the child process terminates.
  \item When the child process terminates, it returns an exit code to the parent process using the exit() system call. The exit code indicates whether the child process completed successfully or encountered an error.

\end{enumerate}

Overall, the process creation process in an operating system involves creating a new process from an existing one, allowing the new process to execute independently, and managing the communication between the parent and child processes. The C programming language provides system calls such as fork() and exec() to facilitate this process.


\section{Process Creation}
\textbf{Aim :} To write a program to create a process in UNIX.

\textbf{Algorithm :}
\begin{list}{}{}
  \item STEP 1: Start the program. 
  \item STEP 2: Declare pid as integer.
  \item STEP 3: Create the process using Fork command.
  \item STEP 4: Check pid is less than 0 then print error else if pid is equal to 0 then execute
  command else parent process wait for child process.
  \item STEP 5: Stop the program.
\end{list}


\textbf{Program :}
\begin{verbatim}
  #include <stdio.h>
  #include <sys/types.h>
  #include <unistd.h>

  int main(){
      pid_t pid, mypid, myppid;
      pid = getpid();

      printf("Before fork: Process id is %d\n", pid);
      pid = fork();
      printf("After fork: Process id is %d\n", pid);

      if(pid < 0){
          perror("fork() failure \n");
          return 1;
      }
      
      if(pid == 0){
          printf("This is child process\n");
          mypid = getpid();
          myppid = getppid();
          printf("Process id is %d and PPID is %d\n", mypid, myppid);
      }
      else{
          sleep(2);
          printf("This is parent process\n");
          mypid = getpid();
          myppid = getppid();
          printf("Process id is %d and PPID is %d \n", mypid, myppid);
          printf("Newly created process id or child pid is %d\n", pid);
      }
      return 0;
  }
\end{verbatim}

\textbf{Output :}
\begin{verbatim}
  Before fork: Process id is 32828
  After fork: Process id is 0
  This is child process
  Process id is 32838 and PPID is 32828
  After fork: Process id is 32838
  This is parent process
  Process id is 32828 and PPID is 32813 
  Newly created process id or child pid is 32838
\end{verbatim}




\section{Executing a command}
\textbf{Aim : } To write a program for executing a command.

\textbf{Algorithm : }
\begin{list}{}{}
  \item STEP 1 :Start the program.
  \item STEP 2: Execute the command in the shell program using exec ls.
  \item STEP 3: Stop the execution.
\end{list}

\textbf{Program : }
\begin{verbatim}
  echo "Program for executing UNIX command using shell programming "
  echo "Welcome"
  ps
\end{verbatim}

\textbf{Output:}
\begin{verbatim}
  Program for executing UNIX command using shell programming 
  Welcome
      PID TTY          TIME CMD
    33095 pts/4    00:00:00 bash
    33110 pts/4    00:00:00 bash
    33111 pts/4    00:00:00 ps  
\end{verbatim}


\section{Sleep Command}
\textbf{Aim : } To create child with sleep command.

\textbf{Algorithm : }
\begin{list}{}{}
  \item STEP 1: Start the program.
  \item STEP 2: Create process using fork and assign into a variable.
  \item STEP 3: If the value of variable is less than zero : print not create and greater than 0 :  process create
  and else print child create.
  \item STEP 4: Create child with sleep of
  2. 
  \item STEP 5: Stop the program.
\end{list}

\textbf{Program : }
\begin{verbatim}
  #include <stdio.h>
  #include <unistd.h>
  #include <stdlib.h> // for exit(0) command

  int main(){
      int pid = fork();
      if (pid < 0){
          printf("Process not created");
          return -1;
      } 
      if (pid == 0){
          sleep(2);
          printf("Child process created");
      }
      else{
          printf("Parent Process created");
      }
  }
\end{verbatim}

\textbf{Ouput :}
\begin{verbatim}
  Parent Process created
  Child process created
\end{verbatim}

\section{Sleep Command Using GETPID}
\textbf{Aim : } To create child with sleep command using getpid.

\textbf{Algorithm : }
\begin{list}{}{}
  \item STEP 1: Start the execution and create a process using fork() command.
  \item STEP 2: Make the parent process to sleep for 10 seconds.
  \item STEP 3:In the child process print it pid and it corresponding pid.
  \item STEP 4: Make the child process to sleep for 5 seconds.
  \item STEP 5: Again print it pid and it parent pid.
  \item STEP 6: After making the sleep for the parent process for 10 seconds print it
  pid.
  \item STEP 7: Stop the execution.
\end{list}

\textbf{Program : }
\begin{verbatim}
  #include <stdio.h>
  #include <unistd.h>
  
  int main(){
      int ppid;
      int pid = fork();
      if (pid < 0){
          printf("Process not created\n");
          return -1;
      }
      if (pid > 0){
          printf("Parent process\n");
          sleep(10);
          pid = getpid();
          printf("Parent process id is %d\n", pid);
      }
      else{
          printf("Child process\n");
          pid = getpid();
          ppid = getppid();
          printf("child process id %d and parent process id is %d\n", pid, ppid);
          sleep(5);
          printf("Child process after 5s sleep \n");
          printf("child process id %d and parent process id is %d\n", pid, ppid);
      }
  }
\end{verbatim}

\textbf{Output : }
\begin{verbatim}
  Parent process
  Child process
  child process id 33852 and parent process id is 33841
  Child process after 5s sleep 
  child process id 33852 and parent process id is 33841
  Parent process id is 33841
\end{verbatim}



\section{Signal Handling}
\textbf{Aim : } To write a program for signal handling in UNIX.

\textbf{Algorithm : }
\begin{list}{}{}
  \item STEP 1:start the program
  \item STEP 2:Read the value of
  pid.
  \item STEP 3:Kill the command surely using kill-9 pid.
  \item STEP 4:Stop the program.
\end{list}


\textbf{Program : }
\begin{verbatim}
  echo "Program for performing KILL operations"
  ps
  echo "enter the pid"
  read pid
  kill -9 $pid
  echo "finished"  
\end{verbatim}

\textbf{Ouput : }
\begin{verbatim}
  Program for performing KILL operations
      PID TTY          TIME CMD
    34080 pts/4    00:00:00 bash
    34112 pts/4    00:00:00 bash
    34113 pts/4    00:00:00 ps
  enter the pid
  34112
  Killed
\end{verbatim}

\section{Wait Command}
\textbf{Aim : } To perform wait command using c program.

\textbf{Algorithm : }
\begin{list}{}{}
  \item STEP 1: Start the execution
  \item STEP 2: Create process using fork and assign it to a variable
  \item STEP 3: Check for the condition pid is equal to 0
  \item STEP 4: If it is true print the value of i and terminate the child process
  \item STEP 5: If it is not a parent process has to wait until the child terminate
  \item STEP 6: Stop the execution
\end{list}


\textbf{Program : }
\begin{verbatim}
  #include <stdio.h>
  #include <unistd.h>
  
  #include <sys/types.h>
  #include <sys/wait.h>
  
  int main(){
      int pid, ppid;
      pid = fork();
      if(pid < 0){
          printf("Process not created");
          return -1;
      }
      if (pid == 0){
          printf("Child Process \n");
          pid = getpid();
          printf("Child process id is %d\n", pid);
          return 0;
      }
      else{
          wait(NULL);
          printf("Parent Process executing\n");
      }
      return 0;
  }
\end{verbatim}

\textbf{Ouput : }
\begin{verbatim}
  Child Process 
  Child process id is 34400
  Parent Process executing
\end{verbatim}

\section{Conclusion}
In this lab, we created process using C program fork method. Then we studied various cases  by creating the parent and child process. Experiment using the sleep and wait command was performed. Moreover, we created bash program to kill the process and see what process are currently running. Hence, in this lab we studied and implemented basic of process creation in an operating system using C programming and shell programming successfully.


\end{document}